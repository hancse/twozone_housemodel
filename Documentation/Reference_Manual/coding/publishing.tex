\section{Publishing a Python package and Docs}
\label{appendix:third}

% definition of boxes
\definecolor{main}{HTML}{5989cf}    % setting main color to be used
\definecolor{sub}{HTML}{cde4ff}     % setting sub color to be used

%\tcbset{
%	sharp corners,
%	colback = white,
%	before skip = 0.2cm,    % add extra space before the box
%	after skip = 0.5cm      % add extra space after the box
%}                           % setting global options for tcolorbox

\definecolor{codegreen}{rgb}{0,0.6,0}
\definecolor{codegray}{rgb}{0.5,0.5,0.5}
\definecolor{codepurple}{rgb}{0.58,0,0.82}
\definecolor{backcolour}{rgb}{0.95,0.95,0.92}

\lstdefinestyle{mystyle}{
	backgroundcolor=\color{backcolour},   
	commentstyle=\color{codegreen},
	keywordstyle=\color{magenta},
	numberstyle=\tiny\color{codegray},
	stringstyle=\color{codepurple},
	basicstyle=\ttfamily\footnotesize,
	breakatwhitespace=false,         
	breaklines=true,                 
	captionpos=t,                    
	keepspaces=true,                 
	numbers=left,                    
	numbersep=5pt,                  
	showspaces=false,                
	showstringspaces=false,
	showtabs=false,                  
	tabsize=2
}

\lstdefinestyle{DOS}
{
	backgroundcolor=\color{black},
	basicstyle=\footnotesize\color{white}\ttfamily
	identifierstyle=\color{white}
	commentstyle=\color{white},
	keywordstyle=\color{white},
	stringstyle=\color{white},
	frame=lines,
	keepspaces=true,
	numbers=none,
	aboveskip=0\baselineskip
}

\subsection{Converting Python code into Python packages}

Once your code exceeds the complexity of writing simple Python \emph{scripts} or \emph{notebooks} that perform a calculation and present the results, it is time to think about code structure. Python code can be divided in \emph{modules} (\textbf{*.py} files). Each module may contain a number of \emph{functions} that belong together. Data structures and dedicated data manipulation functions can be organized in \emph{classes} (objects). A class usually has its own Python module. The programming style then evolves to \emph{object-oriented programming} (OOP). Classes and helper functions that operate together may form a larger framework: a Python \emph{package}.

The advantage of casting your code in the form of a Python package is, that it can be \emph{re-used} by yourself, team members, colleagues or any other interested programmer. Once the package code enters a stable state of development, it can be documented, published and installed. Thus it becomes a reliable building block for new software, reducing complexity in a new application.

In order to achieve a good starting point for \emph{packaging}, the Python code has to be organized in a structured manner. The most recent insights for organizing your code recommend to follow the directory tree structure presented below: \\

\dirtree{%
	.1 \textit{<repo\_root>(-git)}.
	.2 \textbf{docs}.
	.2 \textbf{src}.
	.3 \textit{<package>}.
	.4 \_\_init\_\_.py.
	.4 \textit{<package\_modules>}.py.
	.4 ....
	.4 \textit{<subpackage>}.
	.5 \_\_init\_\_.py.
	.5 \textit{<subpackage\_modules}.py.
	.4 ....
	.2 \textbf{tests}.
	.3 \_\_init\_\_.py.
	.3 test*.py.
	.3 ....
	.2 (.gitignore).
	.2 .readthedocs.yaml.
	.2 environment.yml.
	.2 LICENSE.
	.2 pyproject.toml.
	.2 README.md or README.rst.
	.2 setup.py.
}

\textbf{Notes:} 
\begin{itemize}
	\item Previous recommendations included having the \textit{\textless package\textgreater} folder directly under the \textit{\textless repo\_root\textgreater} folder \textit{i.e.} excluding the \textsf{src} subfolder. This works for smaller packages. See: \url{https://packaging.python.org/en/latest/discussions/src-layout-vs-flat-layout/} and 
	\item every \textit{\textless package\textgreater} and \textit{\textless subpackage\textgreater} folder as well as the \textsf{tests} folder need an \textsf{\_\_init\_\_.py} file. This file can be empty.
	\item some sources recommend having a \textsf{tests} folder under the \textit{\textless package\textgreater} folder. The website \url{https://docs.pytest.org/en/7.1.x/explanation/goodpractices.html} summarizes the options that work.
\end{itemize}

\subsection{The \textsf{conda} environment}

During the development of the code for your package, you have used \textsf{import} statements for common dependencies in most Python modules. These dependencies have been installed in a dedicated \textsf{conda} environment with the commands \textsf{conda install} and \textsf{pip install}. For keeping track of how the environment is built, there are two approaches:

\begin{itemize}
	\item record every \textsf{conda} command needed for building the environment in a text file \textsf{environment.txt}, located in the \textit{\textless repo\_root\textgreater}. For regenerating the environment, use the console commands:
\begin{lstlisting}[style=DOS]
	cd <repo_root>
	source environment.txt
\end{lstlisting}
    \item generate a YAML-file \textsf{environment.yml} in the \textit{\textless repo\_root\textgreater} which can be executed by the command
\begin{lstlisting}[style=DOS]
    cd <repo_root>
    conda env create -f environment.yml
\end{lstlisting}

    For details about the format of the \textsf{environment.yml} configuration file, consult \url{https://docs.conda.io/projects/conda/en/latest/user-guide/tasks/manage-environments.html#creating-an-environment-from-an-environment-yml-file}. An alternative manual can be found at \url{https://medium.com/@balance1150/how-to-build-a-conda-environment-through-a-yaml-file-db185acf5d22}
\end{itemize}

Publishing your package involves documenting your code, so that future users can consult a manual after installing your software. The publishing process starts with activating your (development) environment in the Anaconda Prompt (Windows) or in a Bash shell (Linux):

\begin{lstlisting}[style=DOS]
	activate <your-development-environment>
\end{lstlisting}

\textbf{Note:} The documentation generator \textsf{sphinx} has to operate from the development environment of the package to be published, because it reads import statements in the modules. If sphinx is run from a different environment it may issue annoying warnings about not being able to find imported (standard) packages like numpy, scipy, yaml etc.

\subsection{Documenting a Python package with Sphinx}

\subsubsection{Getting started}

In the \textsf{conda} environment that was created for the development of the package, install the \textsf{sphinx} package and check the installation of \textsf{sphinx-build}. Navigate to the \textit{\textless repo\_root\textgreater} folder of your Python code. From this folder, run \textsf{sphinx-quickstart}, with "docs" as argument. This will create a "\textsf{docs}" subfolder (as customary) and initialize the \textsf{Sphinx} documentation:

\begin{lstlisting}[style=DOS]
	conda install sphinx
	sphinx-build --version
	
	cd <repo_root>
	sphinx-quickstart docs	
\end{lstlisting}

\textsf{sphinx-quickstart} will present to you a series of questions required to create the basic directory and configuration layout for your project inside the \textsf{docs} folder. To proceed, answer each question as follows:
\begin{itemize}
	\item Separate source and build directories (y/n) [n]: \textbf{y}
	\item Project name: enter package name of your code.
    \item Author name(s): enter your name.
	\item Project release [ ]: enter 0.0.1 or any real version number.
	\item Project language [en]: press Enter, unless you really want to document your code in an other language.
\end{itemize}

After the last question, you will see the new docs directory with the following content. \\

\dirtree{%
	.1 docs.
	.2 build.
	.2 make.bat.
	.2 Makefile.
	.2 source.
	.3 conf.py.
	.3 index.rst.
	.3 \_static.
	.3 \_templates.
}\label{tree:src} 

The purpose of each of these files is:

\begin{itemize}
	\item \textsf{build/}: An empty directory (for now) that will hold the rendered documentation.
	\item \textsf{make.bat} and\textsf{ Makefile}: Convenience scripts to simplify some common Sphinx operations, such as rendering the content.
	\item \textsf{source/conf.py}: A Python script holding the configuration of the Sphinx project. It contains the project name and release you specified to sphinx-quickstart, as well as some extra configuration keys.
	\item \textsf{source/index.rst}: The root document of the project, which serves as welcome page and contains the root of the “table of contents tree” (or toctree).
	
\end{itemize}

Generate your first documentation website using the commands:
\begin{lstlisting}[style=DOS]
	cd docs
	make html
\end{lstlisting}

Open \textsf{docs/build/html/index.html} in your browser by double clicking. You should see a basic version of your documentation webpage.
For a summary, see \url{https://www.sphinx-doc.org/en/master/tutorial/getting-started.html}.

\subsubsection{Customization}

To make the documentation look nicer and to include the docstrings in your Python code, a few extra steps have to be taken. Have a look at the website \url{https://sphinx-tutorial.readthedocs.io/start/} under \textbf{Custom Theme}. This website uses the theme \textsf{sphinx\_rtd\_theme}.

\begin{lstlisting}[style=DOS]
	pip install sphinx-rtd-theme
\end{lstlisting}

\textbf{Note}: The package \textsf{sphinx-rtd-theme} is installed with \textsf{pip} since the \textsf{conda package manager} forces a downgrade of the \textsf{sphinx} package.

\subsubsection{Editing conf.py}

Further customization of the documentation can be achieved by fine-tuning the contents of the configuration file \textsf{conf.py}. After running \textsf{sphinx-quickstart} the default \textsf{conf.py} looks like this:

\lstinputlisting[label=pub:conf,caption=Sphinx configuration file conf.py, language=Python, linerange={1-50}]{../../docs/source/conf.py}

First, the documentation generator should be able to find the \emph{package} in the repository. A \emph{relative} path from the directory where \textsf{conf.py} is located should be provided. Before the section \textsf{Project information} insert a section \textsf{Path setup} as follows:

\begin{lstlisting}[style=DOS]
	# -- Path setup --------------------------------------------------------------
	
	# If extensions (or modules to document with autodoc) are in another directory,
	# add these directories to sys.path here. If the directory is relative to the
	# documentation root, use os.path.abspath to make it absolute, like shown here.
	#
	import os
	import sys
	# sys.path.insert(0, os.path.abspath('../../src')) # conf.py in docs\source and <package> in src
	# sys.path.insert(0, os.path.abspath('../../'))    # conf.py in docs\source and <package> in repo_root
	# sys.path.insert(0, os.path.abspath('../src'))     # conf.py in docs and <package> in src
    # sys.path.insert(0, os.path.abspath('../'))      # conf.py in docs and <package> in repo_root
\end{lstlisting}

In some versions of \textsf{sphinx-quickstart} this section is already present as a template. In this case, uncomment and change the line starting with \textsf{sys.path.insert...} according to the guidelines above.

Next, in the section \textsf{General configuration} add the necessary extensions:

\begin{lstlisting}[style=DOS]
    # Add any Sphinx extension module names here, as strings. They can be
    # extensions coming with Sphinx (named 'sphinx.ext.*') or your custom
    # ones.
    extensions = [
        'sphinx.ext.autodoc',
        'sphinx.ext.mathjax',
        'sphinx.ext.viewcode',
        'sphinx.ext.napoleon',
    ]
    
    napoleon_google_docstring = True
    napoleon_use_param = False
\end{lstlisting}

Consult: 
\url{https://www.sphinx-doc.org/en/master/usage/extensions/autodoc.html} \\
https://www.sphinx-doc.org/en/master/usage/extensions/napoleon.html

The \textsf{sphinx.ext.napoleon} extension is necessary for reading and converting Google-style docstrings from your code in \textsf{Sphinx}. The two parameters shown have to be set for a straightforward result.

After the line "templates\_path = ['\_templates']" insert:
\begin{lstlisting}[style=DOS]
    # The suffix(es) of source filenames.
    # You can specify multiple suffix as a list of string:
    #
    # source_suffix = ['.rst', '.md']
    source_suffix = '.rst'
\end{lstlisting}

This enables the option to include only the (official) \textsf{sphinx} file format \textsf{*.rst} or to also parse \textsf{*.md} (markdown) files.

After that, insert the exclusion patterns. If you did not choose separate source and build directories, exclusion of \textsf{\_build} is mandatory. Thumbnail files and MacOS bookkeeping files can be excluded, for portability:

\begin{lstlisting}[style=DOS]
	# List of patterns, relative to source directory, that match files and
	# directories to ignore when looking for source files.
	# These patterns also affect html_static_path and html_extra_path
	exclude_patterns = ['_build', 'Thumbs.db', '.DS_Store']
\end{lstlisting}

Replace the section \textsf{Options for HTML output} by the following snippet. You can choose the \textsf{html\_theme} by uncommenting the required option. For many applications, the standard 'alabaster' theme is replaced by the 'sphinx\_rtd\_theme', which is familiar from the 'ReadTheDocs' websites \url{https://readthedocs.org/}.

\begin{lstlisting}[style=DOS]
	# The theme to use for HTML and HTML Help pages.  See the documentation for
	# a list of builtin themes.
	# html_theme = 'alabaster'
	html_theme = 'sphinx_rtd_theme'
	
	# Add any paths that contain custom static files (such as style sheets) here,
	# relative to this directory. They are copied after the builtin static files,
	# so a file named "default.css" will overwrite the builtin "default.css".
	html_static_path = ['_static']
\end{lstlisting}

\subsubsection{Including docstrings in Sphinx documentation}

Automatic inclusion of the docstrings in your code is initialized by running the program \textsf{sphinx-autodoc}. This program is included in the \textsf{Sphinx} package. The command is run from the \textsf{docs} directory. The output files \textsf{\textless package\textgreater.rst} and \textsf{modules.rst} are added to the \textsf{docs/source} directory. The second argument points to the (input) \textless package\textgreater folder. Choose one of the options, dependent on where the \textless package\textgreater folder is located.

\begin{lstlisting}[style=DOS]
    cd docs/
    sphinx-apidoc -o source/ ../<package>
    sphinx-apidoc -o source/ ../src/<package>
\end{lstlisting}

\textsf{sphinx-apidoc} has a few extra options. See: \url{https://www.sphinx-doc.org/en/master/man/sphinx-apidoc.html} and \url{https://github.com/sphinx-contrib/apidoc}. \\

\subsubsection{Making a ReadTheDocs site}
Finally, make a website with source code documentation in ReadTheDocs format:

\begin{lstlisting}[style=DOS]
	cd docs
	make html
\end{lstlisting}

Alternatively, automate the website generation with \textsf{sphinx-autobuild}:

\begin{lstlisting}[style=DOS]
	conda install sphinx-autobuild
	cd ..              # to repo root
	sphinx-autobuild docs docs/build/html  # repeat (from repo root)
\end{lstlisting}

The website is then automatically hosted on \url{ http://127.0.0.1:8000}. See \url{https://pypi.org/project/sphinx-autobuild/}. 

\subsection{Packaging and Uploading to PyPi}

Once the documentation is presentable, it will be possible to write a summary in a file \textsf{README.rst}. This file has to be located in the \textsf{<repo\_root>} folder. In some cases, this README.rst file is a page from the documentation. Some developers first make a README.rst, and then include it in the ReadTheDocs documentation.

To make things a bit complicated, many packages contain a \textsf{README.md} in the root. This \textit{Markdown}-formatted file is the default format for uploading to PyPi. In such cases, a translation has to be made for Sphinx using the program \textsf{pandoc}. Fortunately, GitHub, BitBucket and PyPi nowadays can handle both *.rst and *.md formats.

Note: You can have either README.rst or README.md in your root folder. Having both is not recommended. PyPi gives an error when uploading the package in such cases.

The packaging process is described in \url{https://packaging.python.org/en/latest/tutorials/packaging-projects/}. The first step is building a \emph{distribution}. Install the package \textsf{build}:

\begin{lstlisting}[style=DOS]
    pip install --upgrade build
      or
    conda install build
\end{lstlisting}

As mentioned on the webpage, the \textsf{build} command needs a configuration file \textsf{pyproject.toml}. A minimal example is given there. If this file is present in the \textless repo\_root\textgreater, two distribution packages will be generated with the command:  

\begin{lstlisting}[style=DOS]
    python -m build	
\end{lstlisting}

The source distribution \textsf{*.tar.gz} and the built distribution \textsf{*.whl} will be created in the \textsf{dist} folder. A \textsf{dist} folder is created automatically if not present.

\subsubsection{Uploading to TestPyPi}

Uploading the distributions to PyPi is done with the command \textsf{twine}. Download the package with:

\begin{lstlisting}[style=DOS]
	pip install --upgrade twine
	  or
	conda install twine
\end{lstlisting}

Once installed, run Twine to upload all of the archives under dist:

\begin{lstlisting}[style=DOS]
    python -m twine upload --repository testpypi dist/*
\end{lstlisting}

Once uploaded, your package should be viewable on TestPyPI; for example: \\ 
https://test.pypi.org/project/\textless package\textgreater.

Note: the program \textsf{twine} asks for an accountname and password. There, it exhibits peculiar behaviour:
\begin{itemize}
	\item copying a password with ctrl-C after the prompt results in an error message.
	\item pasting a password using the command menu from the Windows Command Prompt works.
	\item by the end of 2023, (Test)PyPi requires users to login with 2-factor authorization (2FA). It seems that, if you activate 2FA, uploading code is ONLY possible with username "\_\_token\_\_" and an "API token" as password. The personal username/password is only valid for logging in to the (Test)PyPi site. Once logged in you can generate an API token.  See: \href{https://pypi.org/help/#apitoken}{PyPi API token}.
\end{itemize} 

TestPyPi does not allow updating a package. For each subsequent upload of the distribution files the version number in \textsf{pyproject.toml} has to be changed (increased). This means that your local \textsf{dist} folder becomes crowded with different versions (version pairs) of the distribution files. Every next time you upload the distribution, you either have to:
\begin{itemize}
	\item delete the previous versions manually.
	\item upload only the most recent version of the distribution files. The command is:
\end{itemize} 

\begin{lstlisting}[style=DOS]
	twine upload --repository testpypi --skip-existing dist/*
\end{lstlisting}

\subsubsection{Uploading to PyPi}

Uploading "\emph{for real}" to Pypi is done in a similar way as uploading to the testsite:

Run \textsf{twine} to upload all of the archives under dist:

\begin{lstlisting}[style=DOS]
	python -m twine upload --repository pypi dist/*
\end{lstlisting}

Once uploaded, your package should be viewable on PyPI: it is out in the open now! \\
Example: \\ 
https://test.pypi.org/project/\textless package\textgreater.


PyPi allows updating a particular version of a package. However, this should only be done in case of a clear mishap or mistake. In general, any significant change should at least update the minor version number or build number. Remember to make these changes manually in \textsf{pyproject.toml}. If your local \textsf{dist} folder becomes crowded with different versions (version pairs) of the distribution files, either perform:
\begin{itemize}
	\item delete the previous versions manually.
	\item upload only the most recent version of the distribution files. The command is:
\end{itemize} 

\begin{lstlisting}[style=DOS]
	twine upload --repository pypi --skip-existing dist/*
\end{lstlisting}

\subsubsection{Uploading to Anaconda}

\subsection{Remarks}

\subsubsection{"flat" layout or "src" layout}

The proposed structure for the repository is shown in \ref{tree:src}. The package itself is "encapsulated" in a \textsf{src} folder.
This has become the "src" layout for a package repository. Previously, the "flat" layout, with the package located directly under the \textless repo\_root\textgreater, was more common.

The underlying philosophy is discussed extensively. The main reasons for choosing the "src" layout is:
\begin{itemize}
	\item it is only possible to run tests (from the \textsf{tests} folder) against the \emph{installed package}, not against the package source. The package is not \emph{importable} from the \textless repo\_root\textgreater.
	\item the installed package only contains the modules from the \textsf{src} folder. In the flat layout, commands like "\textsf{import setup.py}" are successful, which is not meant to be.
\end{itemize}
Also see: \\
\url{https://packaging.python.org/en/latest/discussions/src-layout-vs-flat-layout/} \\
\url{https://hynek.me/articles/testing-packaging/} \\
\url{https://blog.ionelmc.ro/2014/05/25/python-packaging/} \\


\begin{lstlisting}[style=DOS]
	cd <repo\_root>
	pip install -e .
\end{lstlisting}


\newpage

\url{https://packaging.python.org/en/latest/tutorials/packaging-projects/} +++

\url{https://stackoverflow.com/questions/72712965/does-the-src-folder-in-pypi-packaging-have-a-special-meaning-or-is-it-only-a-co}

\url{https://stackoverflow.com/questions/49589082/use-pytest-within-a-python-project-using-the-src-layout}

\url{https://medium.com/mlearning-ai/a-practical-guide-to-python-project-structure-and-packaging-90c7f7a04f95}

\url{https://www.pyopensci.org/python-package-guide/package-structure-code/python-package-structure.html}

\url{https://hynek.me/articles/testing-packaging/}

\url{https://medium.com/analytics-vidhya/explicit-understanding-of-python-package-building-structuring-4ac7054c0749}

\url{https://blog.ionelmc.ro/2014/05/25/python-packaging/}

\url{https://github.com/pypa/sampleproject}

\url{https://realpython.com/python-modules-packages/}

\url{https://realpython.com/python-application-layouts/}

\url{https://realpython.com/pypi-publish-python-package/}

\url{https://py-pkgs.org/04-package-structure.html}

\url{https://ianhopkinson.org.uk/2022/02/
	understanding-setup-py-setup-cfg-and-pyproject-toml-in-python/}

\url{http://ivory.idyll.org/blog/2021-transition-to-pyproject.toml-example.html}

\url{https://snarky.ca/clarifying-pep-518/}

\url{https://godatadriven.com/blog/a-practical-guide-to-setuptools-and-pyproject-toml/}

\url{https://docs.python-guide.org/writing/structure/}

\url{https://python.land/project-structure/python-packages}

\url{https://docs.python.org/3/tutorial/modules.html}

\url{https://packaging.python.org/en/latest/overview/}

\url{https://packaging.python.org/en/latest/guides/using-testpypi/}

\url{https://www.sphinx-doc.org/en/master/usage/configuration.html}

\url{https://www.sphinx-doc.org/en/master/usage/quickstart.html}

\url{https://towardsdatascience.com/documenting-python-code-with-sphinx-554e1d6c4f6d}

\url{https://github.com/readthedocs-examples/example-sphinx-basic/blob/main/docs/conf.py}
\url{https://example-sphinx-basic.readthedocs.io/en/latest/index.html}

\url{https://pythonhosted.org/an_example_pypi_project/sphinx.html} 005

\url{https://stackoverflow.com/questions/8218039/how-to-add-extra-whitespace-between-section-header-and-a-paragraph}

\url{https://www.devdungeon.com/content/restructuredtext-rst-tutorial-0}

\url{https://stackoverflow.com/questions/29221551/can-sphinx-napoleon-document-function-returning-multiple-arguments/29343326#29343326}

\newpage

